\documentclass[preprint]{iacrtrans}
\usepackage[utf8]{inputenc}
\usepackage{titlesec}

% Select what to do with todonotes: 
% \usepackage[disable]{todonotes} % notes not showed
\usepackage[draft,color=orange!20!white,linecolor=orange,textwidth=3cm,colorinlistoftodos]{todonotes}   % notes showed
\setlength{\marginparwidth}{3cm}
\usepackage{graphicx}
\usepackage{soul}
\graphicspath{{images/}} % end dirs with `/'
% \usepackage{longtable}
\usepackage{tikz}
\usetikzlibrary{arrows}
\usetikzlibrary{arrows.meta}
\usetikzlibrary{positioning}
\usetikzlibrary{calc}
\usetikzlibrary{backgrounds}
\usetikzlibrary{arrows}
\usetikzlibrary{crypto.symbols}
\tikzset{shadows=no}        % Option: add shadows to XOR, ADD, etc.

\author{Ella Rose\inst{1}}
\institute{Paso Robles, CA \email{python_pride@protonmail.com}}
\title[\texttt EPQ Key Exchange]{\texttt EPQ Key Exchange}

\begin{document}

\maketitle

% use optional argument because the \LaTeX command breaks the PDF keywords
\keywords[Key Exchange]{Key Exchange}\todo{Keywords?}

\begin{abstract}
  We define an algorithm for performing a key exchange between two parties. The algorithm makes use of a small number of bignum multiplications and additions and is conceptually simple.\\ 
\end{abstract}

\todototoc
\listoftodos

\section{Introduction}
 The ability to exchange cryptographic keys over insecure channels is of vital importance for secure communications in general; Without algorithms for arranging secure communication online, much of every day life in the modern world would cease to be possible.

The introduction of Quantum Computing and post-quantum cryptography has lead to the development of lattice based cryptographic schemes that are believed to be secure even when faced with quantum parallelism. 

Lattices are also very attractive because they offer homomorphic properties, which is another of the most important topics in cryptography at the moment. This property is important because it allows for the construction of simple public key encryption schemes and key exchange shemes from secret key ciphers. We present such a scheme in this paper, as well as a homomorphic cipher suitable for instantiating the scheme.

\section{Definitions}
We use $+$ to denote integer addition, $ab$ to denote integer multiplication, and $\mod p$ to denote the modulo operation. We use $log(x)$ to indicate the binary logarithm of $x$.

\section{Algorithm}
First, we shall define the asymmetric key exchange algorithm; Then, we shall define a symmetric secret key cipher suitable for instantiating the key exchange algorithm.

\subsection{Key Exchange Algorithm}
Let $p_1$ and $p_2$ be two encryptions of $0$ under the homomorphic secret key cipher. These two values constitute a public key. To share a secret with the owner of the public key, compute:
\begin{align}
    e \gets secret \\
    c \gets (p_1  r_1) + (p_2 r_2) + e
\end{align}

Where $r_1$ and $r_2$ are randomly generated values of appropriate size. Then the value $c$ is shared with the owner of the public key. The owner of the public key can recover $e$ from $c$ by applying the homomorphic decryption operation of the secret key cipher to c. Because $p_1$ is an encryption of $0$, $p_1 r_1 \equiv 0$, similar for $p_2$. So when the decryption operation is applied, the result is $0 + 0 + e \equiv e$.

\subsubsection{Additional Information for Key Generation}
If the size of the shared secret is to be 256 bits, then the ciphertexts produced by symmetric cipher must support the homomorphic addition of a plaintext 256-bit value. 

Not all homomorphic constructions support the addition of plaintext values to ciphertext values; Lattices offer this required feature.

It is important that $p_1$ and $p_2$ are coprime to each other. Otherwise, some bits of the secret may be recovered by reducing a ciphertext modulo the greatest common divisor. Specifically:

\begin{align}
    e \gets secret\\
    c \gets (p_1 r_1) + (p_2 r_2) + e\\
    e_0 \gets c \mod gcd(p_1, p_2)
\end{align}

Where $log(e_0) \equiv log(gcd(p_1, p_2))$. If $p_1$ and $p_2$ are not coprime, then a simple modulo operation will yield $log(gcd(p_1, p_2))$ bits of the secret $e$.

\subsection{Secret Key Cipher}
Creation of the $p_1$ and $p_2$ values requires a secure, additively homomorphic cipher. We define a simple lattice based cipher which utilizes effectively the same operations as the key exchange process. Let $s_1$ and $s_2$ be two large numbers, where $log(s_1) >= log(s_2) * 3$; These two numbers constitute the secret key for the cipher, which in turn acts as the private key for the key exchange presented previously.\\

To encrypt a ciphertext, compute:

\begin{align}
   c \gets (s_1 r_1) + (s_2 r_2) + m
\end{align}

Where the $r$ values are of appropriate size relative to $s_1$ and $s_2$. 

Decryption is computed as:

\begin{align}
    m \gets (c \mod s_1) \mod s_2
\end{align}

\subsubsection{Additional Information for Key Generation}
$s_1$ and $s_2$ should be coprime, and $log(s_1) >= log(s_2) * 3$. To instantiate the key exchange scheme, $log(s_2) >= log(e)$, where $e$ is the secret to be shared. If $e$ is to be 256 bits, then $log(s_2) >= 256$ and $log(s_1) >= 768$. We have a python implementation where $log(s_1) \equiv 880$ and $log(s_2) \equiv 288$ for a total of about 1168 bits, or 146 bytes. 

\subsection{Note}
While the decryption circuit works via two modulo operations of the base lattice points, this operation does not work for the public key/key exchange portion of the algorithm - The reason is because in the secret key cipher, the two different lattice points are of largely different size; In the key exchange algorithm, both points are roughly the same size, and so the double modulo operation does not function identically.

\subsection{Public Key Randomization}
\todo{Needs more development}
Because the public key consists of malleable homomorphic ciphertexts, it can be randomized without complicating any of the rest of the procedures. The presented scheme can utilize encrypted public keys, and we give two potential use cases for the idea.

First, as an additional layer of security, users who wish to send a message to the owner of the public key can first randomize the public key that they are using. Assuming that the randomized $p_1$ and $p_2$ are in effect unrelated to the original $p_1$ and $p_2$, then observers of the ciphertext do not even have the base points that the ciphertext was created with.

Secondly, this feature allows for the disassociation of the distributor of the public key from the possessor of the public key. For example, assume we have three parties, Alice, Bob, and Charlie. Charlie does not want Alice to communicate with Bob. If Charlie searches Alices computer and he finds Bobs public key, then he will know that Alice has been or intended to communicate with Bob. To counter this, Alice can randomize Bobs public key and store only the randomized key. The randomized key will be functionally equivalent to the public key associated with Bob, with the exception that nobody may associate it with Bob. 

Any homomorphic secret-key instantiated scheme can trivially produce a randomized public key using the secret key to generate a new public key; The scheme presented here allows for the generation of a randomized public key using only the public key. The difference is that in the former case, Bob would have to distribute the randomized key, which would associate it with him (assuming it is signed, for example). In the latter case, Bob has only one key, and Alice can obtain a randomized public key on her own.

\subsubsection{Key Randomization Algorithm}
\begin{align}
    p_n \gets p_1 r_1 - p_2 r_2 + p_1 r_3 - p_2 r_4\\
\end{align}

\section{Metrics}
Private keys are slightly less then 1200 bits or 150 bytes. Public keys are approximately 2300 bits or 280 bytes. Ciphertexts from the key exchange algorithm are about 1500 bits or 185 bytes.

The time required to generate a keypair is about .0019 seconds. Our simple python implementation can perform 8192 key exchanges in slightly less then 1 second. While a C implementation would probably be faster, most of the time is likely spent doing bignum arithmetic, which means that a simple python implementation may be acceptably fast.

\section{Conclusion}
 We define a simple lattice based key exchange algorithm. The algorithm utilizes 4 bignum multiplications and 4 bignum additions to generate a keypair, 2 bignum multiplications and 2 bignum additions per key sent, and 2 modulo operations on bignums to receive a key. Public keys occupy about 2300 bits using current parameter cohices, while private keys require about 1200 bits.
\end{document}

