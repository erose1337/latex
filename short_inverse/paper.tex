\documentclass[preprint]{iacrtrans}
\usepackage[utf8]{inputenc}

% Select what to do with todonotes: 
% \usepackage[disable]{todonotes} % notes not showed
\usepackage[draft,color=orange!20!white,linecolor=orange,textwidth=3cm,colorinlistoftodos]{todonotes}   % notes showed
\setlength{\marginparwidth}{3cm}
\usepackage{graphicx}
\usepackage{soul}
\graphicspath{{images/}} % end dirs with `/'
% \usepackage{longtable}
\usepackage{tikz}
\usetikzlibrary{arrows}
\usetikzlibrary{arrows.meta}
\usetikzlibrary{positioning}
\usetikzlibrary{calc}
\usetikzlibrary{backgrounds}
\usetikzlibrary{arrows}
\usetikzlibrary{crypto.symbols}
\tikzset{shadows=no}        % Option: add shadows to XOR, ADD, etc.

\author{Ella Rose}
\institute{Paso Robles, CA \email{python_pride@protonmail.com}}
\title[Lattice Trapdoor from Short Modular Inverse and Noisy Modulus]{Lattice Trapdoor from Short Modular Inverse and Noisy Modulus}

\begin{document}

\maketitle

% use optional argument because the \LaTeX command breaks the PDF keywords
\keywords[homomorphic encryption, public-key cryptography]{homomorphic encryption, public-key cryptography}

\begin{abstract}

 \end{abstract}

\section{Introduction}

\subsection{Definitions}
$a \leftarrow$ indicates the assignment of a value\\
$a + b$ indicates integer addition\\
$a b$ indicates integer multiplication\\
$a \mod b$ indicates modular reduction\\
$a / b$ indicates integer division\\
$random(size)$ indicates generation of a random $size$-bit integer\\
$inverse(a, b)$ indicates computation of the modular multiplicative inverse of $a$ modulo $b$\\

\section{Short Modular Inverses and Noisy Moduli}
$a^{-1} (a s + e) = s + a^{-1} e \mod q\\$
$log(a^{-1}) + log(e) < log(q) \rightarrow s + a^{-1} e \mod q \equiv s + a^{-1} e\\$
$log(s) < log(a^{-1}) \rightarrow s = s + a^{-1} e \mod a^{-1}\\$
A point $a s + e \mod q$  can be changed to a point $s + a^{-1} e \mod q$ via multiplication with $a^{-1}$. If $a^{-1}$ is smaller then $a$, then the total size of the value will shrink, assuming appropriate sizes for the values involved. With the right parameter choices, a point $a s + e \mod q$ can be changed to the point $s + a^{-1} e$. The former is over finite fields, while the latter is over integers. It is trivial to solve for $s$ once the problem is over integers.

The value $a^{-1}$ can serve as trapdoor information, but it can be trivially recovered from $a$ and $q$. The challenge then, is how to distribute $a, q$ such that nobody can recover the short modular inverse $a^{-1}$ from the pair? 

After having tried many combinations of techniques, the most promising answer appears to be the use of a noisy modulus. Instead of distributing $a$ and $q$, instead $a, q + r$ may be distributed. The addition of the noise $r$ in the modulus prevents the computation of the short inverse $a^{-1}$; Anyone may calculate the inverse of $a \mod q + r$, but this value will most likely be approximately the size of the modulus and not usable to shrink the value.

When working modulo $q$ on a value that was reduced modulo $q + r$, noise will be present in the result. The value of the noise can be precisely determined by $r (x / q)$, where $x$ is the value that was reduced modulo $q + r$. The size of the noise scales quickly as the difference in size between $x$ and $q$ becomes large. 

\section{Public Key Cryptosystem}
A private key consists of a relatively small value $a^{-1}$ and a secret modulus $q$. A public key consists of the modular inverse of $a^{-1}$ modulo $q$ and a noisy version of the modulus $q_r$.  

The public key operation consists of selecting a random point $s$ on the $a$ lattice, adding some large error term $e$, then applying modular reduction by the modulus $q_r$.

The private key operation consists of multiplication by the short inverse $a^{-1}$ modulo $q$, followed by modulo $a^{-1}$. The multiplication by the short inverse changes the value to the point $e$ on the $a^{-1}$ lattice, with an error of $s$. This shrinks the size of the value to where it is effectively over integers, which allows the recovery of $s$ via simple modular reduction by $a^{-1}$.

\subsection{Key Generation}
\begin{flalign*}
a^{-1} \leftarrow random(a^{-1}_{size})\\
q \leftarrow random(q_{size})\\
a \leftarrow modular\_inverse(a^{-1}, q)\\
q_r \leftarrow q + random(r_{size})\\
key_{private} \leftarrow a^{-1},\ q\\
key_{public} \leftarrow a, q_r
\end{flalign*}

\subsection{Public Key Operation}
\begin{flalign*}
s \leftarrow random(s_{size})\\
e \leftarrow random(e_{size})\\
c \leftarrow a s + e \mod q_r
\end{flalign*}

\subsection{Private Key Operation}
\begin{flalign*}
s \leftarrow a^{-1} c \mod q \mod a^{-1}
\end{flalign*}

\subsection{Homomorphism}

\subsection{Parameter Sizes}


\section{Somewhat Homomorphic Secret Key Cryptosystem}
\subsection{Key Generation}

\subsection{Encryption}

\subsection{Decryption}

\subsection{Homomorphism}


\section{Security Analysis}

\section{Metrics}

\section{Open Problems}

\section{Conclusion}


\section*{Acknowledgements}

\end{document}